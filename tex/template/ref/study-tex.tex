% 使用 \framebox 和 \rule 绘制任何长宽的矩形边框
	\framebox[长度]{\rule[0cm][宽度]希望写入的内容(可以是图片)}
% 图片置中
	\centerline{\framebox[长度]{\rule[0cm][宽度]希望写入的内容(可以是图片)}}
% 标注引用文献
	\cite{文献label}
% 引用图片
	\ref{图片label}
% 使用 \input 和 \include 将文件分块
	\section{段落1}
这个段落是用来测试的,没有关系拉~
\subsection{子段落1}
再来一个子段落!
\subsubsection{子子段落1}
我靠,有完没有?!

% 插入图片
	\begin{figure}[h]
	%\centerline{\includegraphics[width=6cm, angle=-90]{fig5.eps}}
	\caption{图片题目}
	\label{fig1}
	\end{figure}
% 控制段间距和缩进
	% 1.无间距,无缩进
	para1\\
	para2
	% 2.无间距,有缩进
	para1

	para2
	% 3.有间距,无缩进
	para1\\

	\noindent para2
	% 4.有间距,有缩进
	para1\\

	para2
% 使用 \newcommand
	\newcommand{\新命令名称}[参数个数]{...#1(参数1)...#2...}
% vim-latex suite commands
	<F5> to insert environment
	<F7> to insert command
	EXX quick environment
	SXX quick sectioning
	<ctrl+j> to jump in <+...+> format
	use :help latex to check latex commands
	use :help latex-suite.tex to check vim-latex suite HOWTO
% cross product and dot product
	\times and \cdot
% change color of fonts:
	\color{blue} we are blue fonts.
% 可调整宽度的表格
	\begin{tabular*}{0.95\textwidth}{rcc}
		\hline
	\end{tabular*}
