\thispagestyle{empty}
\title{\textbf{title}}
%\author{
xzpeter@gmail.com
}
%\date{} % <--- leave date empty
%\maketitle\thispagestyle{empty} %% <-- you need this for the first page

\maketit{文章题目}{
	徐哲~xzpeter@gmail.com,\\
	(中国科学院电工研究所)\\[3pt]
}

\makeabs{摘要}{
\xiaosi
原子力显微镜(atomic force microscope, AFM)是一种具有原子分辨率的表面形貌、电磁性能分析的重要仪器。在表面科学、纳米技术领域、生物电子等领域逐渐发展成为重要的的材料表征工具。在扫描探针显微镜中,探针更起到了十分重要的作用。若采用电解腐蚀等常规方法制作探针,难以形成纳米量级的针尖。而采用MEMS技术则可能在硅片上批量形成尖度达到纳米量级的针尖。本课题旨在调查目前市场上各种原子力显微镜探针的种类及其MEMS制作工艺,并对其进行比较,总结其在国内外的发展现状和方向。
}{AFM, 探针, MEMS工艺}

%\newpage \tableofcontents \newpage

\xiaosi
